%导言区
\documentclass[10pt]{ctexart}%book,report,letter
%一般只有10、11、12磅再往上就没有了
\usepackage{ctex}

\newcommand{\myfont}{\textit{\textbf{\textsf{Fancy Text}}}}%自定义命令

\newcommand\degree{^\circ}

\title{\heiti 杂谈勾股定理}
\author{\kaishu 刘佳林}
\date{\today}

%正文区(文稿区)
\begin{document}
	%字体族设置(罗马字体、无衬线字体、打字体字体)
	\textrm{Roman Family} \textsf{Sans Serif Family} \texttt{Typewritter Family}%命令
	
    \rmfamily Roman Family {\sffamily Sans Serif Family} {\ttfamily Typewritter Family}%{}将文本分组,限定字体的作用范围
    %声明
    
    %字体系列设置(粗细、宽度)
    \textmd{Medium Series} \textbf{Boldface Series}
    
    {\mdseries Medium Series} {\bfseries Boldface Series}
    
    %字体形状(直立、斜体、伪斜体、小型大写)
    \textup{Upright Shape} \textit{Italic Shape}
    \textsl{Slanted Shape} \textsc{Small Caps Shape}%命令
    
    {\upshape Upright Shape} {\itshape Itlatic Shape} {\slshape Slanted Shape} {\scshape Small Caps Shape}%声明
    
    %中文字体
    {\songti 宋体} \quad {\heiti 黑体} \quad{\fangsong 仿宋} \quad {\kaishu 楷书}
    
    中文字体的\textbf{粗体}与\textit{斜体}
    
    %字体大小
    {\ting           Hello}\\
    {\scriptsize     Hello}\\
    {\footnotesize   Hello}\\
    {\small          Hello}\\
    {\normalsize     Hello}\\
    {\large          Hello}\\
    {\Large          Hello}\\
    {\LARGE          Hello}\\
    {\huge           Hello}\\
    {\Huge           Hello}\\%指与normalsize相对的大小
    
    %中文字号设置命令
    \zihao{-0} 你好!%-0表示小初号也可改为5号等 
    
    \myfont   
    
    
    \maketitle
	Hello World!%空行即换行
	
	%here is my big formula
	Let $f(x)$ be defined by the formula
	$$f(x)=3x^2+x-1$$ which is a polynomial of degree 2.
	%$(行内公式)or$$(行间公式)包围的符号称为数学模式
	\begin{equation}
		内容...
	\end{equation}%用于产生一个带编号(x)的行间公式
\end{document}